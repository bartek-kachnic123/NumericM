\documentclass{article}
\usepackage[T1]{fontenc}
\usepackage{amsmath}
\usepackage{graphicx}
\begin{document}

\section{Wprowadzenie}

\paragraph{}

Aby efektywnie wyliczyć rozwiązania macierzy wstęgowej zastosujemy metodę LU. Będzie odpowiednia, ponieważ 
dzięki niej jesteśmy w stanie zapisać taką macierz w tablicy o rozmiarze m x n, gdzie \\
 n - długość najdłuzszej wstęgi czyli diagonali, m - ilość wstęg. \\
\paragraph{}
Wybierzmy macierz: \\
\begin{equation*}
 \medskip
A = \begin{pmatrix}
    1.2 & \frac{0.1}{1} & \frac{0.4}{1^{1}} \\ 
    0.2 & 1.2 & \frac{0.1}{2} & \frac{0.4}{2^{2}} \\ 
    ... & ... & ... & ... & ... & ... & ...  \\
     &&& 0.2 & 1.2 & \frac{0.1}{N-2} & \frac{0.4}{(N-2)^{2}} \\ 
     && & & 0.2  & 1.2 & \frac{0.1}{N-1} \\
    &&&&&0.2&1.2
    \end{pmatrix}
\end{equation*}
oraz
\begin{equation*}
 x = (1, 2,...,N)^T \\
\end{equation*}

\smallskip
\paragraph{}
Układ równań Ay = x oraz wyznacznik A obliczymy za pomocą metody LU, którą zoptymalizujemy usuwając niepotrzebne operacje. \\
W pamięci nasza macierz A jest zapisana jako tablica 4 x N. \\
\medskip
Dla i=1, 2 ... N
\[ arrayA_{3, i-1} = arrayA_{3, i-1} / arrayA_{2, i-1} \]
\[ arrayA_{2, i} = arrayA_{2, i} - (arrayA_{1, i-1} * arrayA_{3, i-1})\]
\[arrayA_{1, i} = arrayA_{1, i} - (arrayA_{3, i-1} * arrayA_{0, i-1}) \]
\[ arrayA_{0, i}-bez\ zmian\]

Wyliczając powyższe wartości otrzymujemy macierz LU\\

\bigskip
\section{Wyniki}


\paragraph{}
Dla N = 100: \\
\begin{equation*}
Det(A) = 78240161.00959387 
\end{equation*}
\\\\\\
Rozwiązania y = \\

[3.28713349e-02 1.33962280e+00 2.06648030e+00 2.82554361e+00 \\
 3.55757172e+00 4.28449287e+00 5.00721018e+00 5.72766400e+00 \\
 6.44661558e+00 7.16455440e+00 7.88177388e+00 8.59846587e+00 \\
 9.31475980e+00 1.00307462e+01 1.07464903e+01 1.14620401e+01 \\
 1.21774318e+01 1.28926932e+01 1.36078460e+01 1.43229071e+01 \\
 1.50378905e+01 1.57528071e+01 1.64676661e+01 1.71824750e+01 \\
 1.78972401e+01 1.86119666e+01 1.93266590e+01 2.00413212e+01 \\
 2.07559563e+01 2.14705671e+01 2.21851562e+01 2.28997256e+01 \\
 2.36142771e+01 2.43288125e+01 2.50433330e+01 2.57578401e+01 \\
 2.64723348e+01 2.71868182e+01 2.79012910e+01 2.86157543e+01 \\
 2.93302086e+01 3.00446546e+01 3.07590929e+01 3.14735241e+01 \\
 3.21879487e+01 3.29023671e+01 3.36167796e+01 3.43311868e+01 \\
 3.50455888e+01 3.57599861e+01 3.64743789e+01 3.71887674e+01 \\
 3.79031520e+01 3.86175328e+01 3.93319101e+01 4.00462841e+01 \\
 4.07606549e+01 4.14750227e+01 4.21893876e+01 4.29037498e+01 \\
 4.36181095e+01 4.43324668e+01 4.50468217e+01 4.57611744e+01 \\
 4.64755250e+01 4.71898736e+01 4.79042203e+01 4.86185652e+01 \\
 4.93329083e+01 5.00472497e+01 5.07615896e+01 5.14759279e+01 \\
 5.21902647e+01 5.29046002e+01 5.36189343e+01 5.43332671e+01 \\
 5.50475986e+01 5.57619290e+01 5.64762582e+01 5.71905863e+01 \\
 5.79049133e+01 5.86192393e+01 5.93335643e+01 6.00478884e+01 \\
 6.07622116e+01 6.14765338e+01 6.21908553e+01 6.29051759e+01 \\
 6.36194957e+01 6.43338147e+01 6.50481330e+01 6.57624506e+01 \\
 6.64767674e+01 6.71910836e+01 6.79053992e+01 6.86197141e+01 \\
 6.93340283e+01 7.00483379e+01 7.07650589e+01 7.15391569e+01] 

\section{Analiza wyników}
\paragraph{}
Za pomocą specjalnych algorytmów jesteśmy w stanie znaleźć rozwiązania układu równań macierzy wstęgowych w czasie O(n).\\
Udaje się nam to osiągnąć pomijając niepotrzebne operacje takie jak mnożenie przez zero i  dodawanie zerowych wartości. \\
Dodatkowo wybrana metoda LU pozwala nam zaoczdzęcić dużo pamięci dzięki zachowaniu takiej samej struktury jak macierz A. \\



\end{document}