\documentclass{article}
\usepackage[T1]{fontenc}
\usepackage{amsmath}
\usepackage{graphicx}
\usepackage{pgf}
\begin{document}
\author{Bartłomiej Kachnic}
\section{Wprowadzenie}

\paragraph{}
\begin{equation}
Naszym celem będzie wyznaczenie wielomianów interpolacyjnych stopnia n, Wn(x) dla przedziału <-1, 1>.
dla funkcji  y(x) = \frac{1}{1 + 25x^2} \end{equation}

\paragraph{}
Dla: \\
\begin{equation*}
jednorodnych węzłów interpolacji x_i = -1 + 2\frac{i}{n+1} dla (i=0, ... , n)
\end{equation*}
oraz
\begin{equation*}
x_i = cos(\frac{2i+1}{2(n+1)}pi)$$

\end{equation*}
\paragraph{}
$a)$ Algorytm QR polega na iteracji: \\
\begin{equation*}
    QR = A 
\end{equation*}
\begin{equation*}
    A = RQ
\end{equation*}
\paragraph{}
$b)$ Metoda potęgowa polega na iteracji: \\
\begin{equation*}
    By_{k} = z_{k}
\end{equation*}
\begin{equation*}
    By_{k+1} = \frac{z_{k}}{||z_{k}||}
\end{equation*}
\medskip
\section{Wyniki}

Zostosowano maksymalną ilośc iteracji N = 100 i błędu przybliżenia 10e-8.
\paragraph{}
Wartości własne macierzy A przy pomocy algorytmu QR:\\
\begin{equation*}
[7.23099229  \ 5.90015728 \ 4.81580659 \  1.05304383]
\end{equation*}

\paragraph{}
Maksymalna wartość własna (co do modułu) macierzy B przy pomocy metody potęgowej\\
\begin{equation*}
10.015982848255206\\
\end{equation*}
\paragraph{}
Odpowiadający jej wektor własny:\\
\begin{equation*}
[0.55829692 \ 0.77620834 \  0.28678783  \ 0.05964813]
\end{equation*}



\section{Analiza wyników}
\paragraph{}
Przy pomocy algorytmu QR jesteśmy w stanie znaleźć wartości własne, jednak koszt wszystkich operacji jest duży ze względu na samą złożonośc faktoryzacji QR , która wynosi $O(n^3)$
Natomiast dzięki metodzie potęgowej znajdujemy maksymalną wartość własną i jej wektor własny. Jest to możliwe tylko wtedy gdy wartości własne przybierają różne wartości modułu.



\end{document}